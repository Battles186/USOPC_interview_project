\documentclass{article}
\usepackage[margin=1.0in]{geometry}
\usepackage{graphicx}
\usepackage{float}
\usepackage{supertabular}
\usepackage{apacite}

\title{USOPC Athlete Wellness and Load Interview Report}
\author{Perry Battles}
\date{May 2025}

\begin{document}

	\maketitle

	\section{Key Takeaways}

		\begin{itemize}
			\item EDA demonstrated that average session heart rate and
			practice load likely express much of the same information.
			If collecting both is logistically intractable or difficult,
			not much is lost by sacrificing one. Because of the familiarity
			of heart rate data and zones, it may be more useful to present
			these values to coaches and athletes.
			\item Sleep has by far the most outsized effect in promoting
			athlete wellness by reducing fatigue and stress and increasing
			mood and motivation.
			\item Workload does not appear to affect sleep quality within
			the ranges in which it has been administered, with the exception
			of a potential slight effect for practice load.
			\item Soreness was negatively associated with sleep to a much
			greater degree than with any other variable.
			Competition load may positively influence soreness.
			\item There was an association between greater
			strength and conditioning load and lower levels of fatigue
			and stress.
			However, this effect is likely very slight and possibly
			confounded by position-based factors; for example, it may be
			the case that positions that engage in more strength and
			conditioning perform less other training.
			\item To promote athlete well-being, facilitating proper
			restful sleep should be considered one of the highest priorities,
			especially during circumstances of travel and international
			competition. Very
			secondarily, the inclusion of resistance training may help to
			reduce levels of fatigue and stress. Finally, since soreness is
			associated with high competition load, techniques to mitigate
			this trend may be beneficial for performance. For example,
			during periods of dense competition frequency when acute
			performance is prioritized over chronic adaptation, utilizing
			recovery modalities that reduce soreness may enhance performance.
		\end{itemize}

		Each of these points is treated much more extensively in the
		extensive report, submitted alongside this document. For a
		more complete treatment of the analytical methods and results,
		please refer to it.

	\section{Introduction and Overview}

		This report is intended to be read by a sports science team to provide
		an overview of key patterns in the data and directions for action
		to facilitate athlete health and wellness and, by extension, performance.

		The first question provoked by the data concerned the position groups;
		since the athletes were aggregated already into different roles, were
		any differences in workload or wellness data simply a function of group?
		This question hasn't been answered extensively enough by the present
		analysis, but my tentative conclusion is that, while there are likely strong
		position-based trends as demonstrated by the exploratory data analysis,
		there are probably also some outlier cases in different positions in which
		a more nuanced model of workload and wellness is useful, and also some
		conclusions that apply across groups.

		The approach I started with was to perform a grid search over several
		different models and hyperparameters in the data. However, once I had
		more carefully inspected the distributions of the wellness outcome
		variables and performed the Shapiro-Wilk test, I concluded that they
		were roughly gamma distributed and warranted analysis using a model
		whose assumptions were reflected by the data. Ultimately, I settled
		on using the gamma generalized linear model, which I implemented in
		R. This led to the insights enumerated above.

	\section{Methods}

		\subsection{Exploratory Data Analysis and Data Preprocessing}

			EDA began by simply visualizing the
			distributions of the data to determine what models and transformations
			might be needed to make inferences with some
			conviction that the assumptions of the analysis are well-met. After
			determining that the data were organized essentially as a time series
			for each athlete/variable combo and were also very sparse, I concluded
			that they should be aggregated on a week-by-week basis to facilitate
			analysis, as trends that are present after grouping are often absent
			when the data are considered only by individual calendar day.
			I also looked at distributions of the variables by position group and
			concluded that there were some marked differences between them.
			However, I chose not to include the athlete's position group
			as a predictor in subsequent modeling approaches; in hindsight, I believe
			that this was correct, since the differences
			between the groups should be reflected in the metrics provided, assuming
			that they contained information that is relevant to the outcomes of
			interest.

		\subsection{Gamma Generalized Linear Modeling}

			Because none of the wellness variables were normally distributed according
			to the Shapiro-Wilk test and appeared to be roughly gamma distributed,
			I moved forward with a gamma generalized linear model,
			which loosens some of the statistical assumptions of standard linear
			regression. I selected some feature subsets that I thought would be
			informative in modeling each outcome based on known physiological
			principles and relationships, and moved forward.

	\section{Challenges}

		The primary challenge I ran into in analyzing these data was the fact that
		there were matrix multiplication errors in the underlying \texttt{NumPy}
		and R \texttt{glm} code that caused fits of the gamma generalized linear model
		to fail. Getting around this took some doing; after reading a bit on
		StackOverflow, it seemed to me that the primary issue was some large numbers
		being produced in one or more steps of the math underlying the model fit.
		After being uncertain how to address this in Python, I pivoted over to using
		R's \texttt{glm} package, initially did not have much luck, but ultimately
		stumbled upon the log linking function as the appropriate choice. This
		allowed the models to converge and get a proper fit.

		Another challenge was that the data do not precisely fit the assumptions
		of many standard models; values that are arbitarily capped at 0 and 100
		are not described very well by many existing distributions. This meant that,
		although the gamma seemed to be a good approximation of the underlying
		distribution, it couldn't be technically correct since a gamma-distributed
		random variable can't actually assume values equal to 0, and won't be
		capped at some arbitrary value. I don't have a ready-made solution to deal
		with that problem, but it is something that could probably be addressed
		by utilizing another form of model, such as a multilayer perceptron or
		similar architecture.

	\section{Conclusions}

		The primary conclusion I've drawn from this analysis is that managing
		workload is tremendously secondary to sleep in facilitating athlete
		performance for most intents and purposes, providing we are operating
		within the confines of reasonable training. If we had to choose between
		carefully trying to monitor athlete loading and attending to
		sleep, we should probably choose to attend to sleep I could see this
		becoming particularly relevant during travel, particularly where
		international competition is concerned.

		I also thought it was very interesting that strength and conditioning
		load was associated with lower levels of stress and fatigue. I'm not
		convinced that this isn't primarily a function of athlete position
		group, since it seems that there is some kind of "keeper" position group
		that doesn't cover nearly as much distance and also probably performs
		more resistance training. In that case, it may be that higher volumes of
		lifting co-occur with lower volumes of other training. Very speculatively,
		it could be that there are even temperamental factors that predispose
		an athlete in this position group to perceive less stress and fatigue.
		The main point here is that this mild trend is present, but it isn't
		clear exactly what action should be taken on the basis of the trend.
		One avenue forward would be to increase the level of strength and conditioning
		modestly across the entire athlete population and see if some beneficial
		change results -- that could move in the direction of answering the
		question of a potential causal relationship there.

	\section{Directions for Future Work}

		Originally, I wanted to have an analytical pipeline written in pure Python
		that performed
		a grid search across several different models and hyperparameter combinations.
		Some of these models (e.g., random forest regressors) don't produce coefficient
		estimates in the traditional sense or may not necessarily make assumptions that
		fit the data very well, but can still be very predictive, possibly even
		more so than the generalized linear model chosen here. Others, like the
		multilayer perceptron, could do a great job predicting values of the
		outcome variable but might not necessarily be very readily interpretable.
		However, there are ways around this: a few solutions that come to mind are
		Shapley values, as well as visualizing outcome variable surfaces when varying
		two inputs at a time while holding the other predictors constant. One can
		also use principal component analysis to find eigenvectors in the inputs,
		reduce the predictor space to be two-dimensional, and then visualize
		changes in the outcome variable that way.

		Another question I would like to more extensively answer is how much of
		the variation in the wellness and workload variables simply depends upon
		player position group. Although this may not change much from a modeling
		standpoint, it could have some important practical applications, since a
		player's position could be a good heuristic for how we need to approach
		managing their workload. That being said, it could be that there are
		underlying player positions that do not correspond to the given groups,
		and discovering those could give us more insight into how to coach these
		players. Manifold learning could be a solution to this problem; it would
		be an easy task to implement this with the present dataset, and it could
		reveal interesting patterns that could improve interventions with athletes.

\end{document}

