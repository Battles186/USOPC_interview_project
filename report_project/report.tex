\documentclass{article}
\usepackage[margin=1.0in]{geometry}
\usepackage{graphicx}
\usepackage{float}
\usepackage{supertabular}
\usepackage{apacite}

\title{USOPC Athlete Wellness and Load Interview Report}
\author{Perry Battles}
\date{May 2025}

\begin{document}

	\maketitle


	\section{Key Takeaways}

		\begin{itemize}
			\item EDA demonstrated that average session heart rate and
			practice load likely express much of the same information.
			If collecting both is logistically intractable or difficult,
			not much is lost by sacrificing one. Because of the familiarity
			with heart rate data and zones, it may be more useful to present
			heart rate data to coaches and athletes.
			\item Sleep has by far the most outsized effect in mitigating
			athlete wellness by reducing fatigue and stress and increasing
			mood and motivation.
			\item Workload does not appear to affect sleep quality within
			the ranges in which it has been administered,
			with the exception of a potential
			slight effect for practice load.
			\item Soreness was negatively associated with sleep to a much
			greater degree than with any other variable.
			Competition load did positively influence soreness.
			\item Athletes that performed more strength and conditioning
			in a given week experienced less stress
			and fatigue.
			However, these effects are likely very slight and possibly
			confounded by the positions of players engaging in the most
			strength and conditioning also performing less other training.
		\end{itemize}

		Each of these points is treated much more extensively in the
		extensive report, submitted alongside this document. For a
		more complete treatment of the analytical methods and results,
		please refer to it.

	\section{Introduction}

		This report is intended to be read by a sports science team to provide
		an overview of key trends in the data and directions for further action
		to facilitate athlete health and wellness and, by extension, performance.

		First glances at the data provoked several interesting questions. Firstly,
		since the athletes were aggregated into different position groups, were
		any differences in workload or wellness data simply a function of group?
		This question hasn't been answered extensively enough by the present
		analysis, but my tentative conclusion is that, while there are likely strong
		position-based trends as demonstrated by the exploratory data analysis,
		there are probably also some outlier cases in different positions in which
		a more nuanced model of workload and wellness is useful, and also some
		conclusions that apply across groups.

	\section{Methods}

		\subsection{Exploratory Data Analysis}

			EDA proceeded in several steps. The first consisted of simply visualizing
			distributions of the data to determine what models and transformations
			might be needed to make inferences on the basis of the data with some
			conviction that the assumptions of the analysis are well-met. From there,
			I looked at distributions of the variables by position group.

		\subsection{Gamma Generalized Linear Modeling}

			Because none of the wellness variables were normally distributed according
			to the Shapiro-Wilk test, I moved forward with a generalized linear model,
			which loosens some of the statistical assumptions of standard linear
			regression. Because the outcome variables seemed to be approximately
			gamma-distributed, I chose the gamma generalized linear model. I selected
			some feature subsets that I thought would be informative in modeling
			each outcome based on known physiological principles and relationships,
			and moved forward.

	\section{Challenges}

		The primary challenge I ran into in analyzing these data was the fact that
		there were matrix multiplication errors in the underlying \texttt{NumPy}
		and R \texttt{glm} code that caused fits of the gamma generalized linear model
		to fail. Getting around this took some doing; after reading a bit on
		StackOverflow, it seemed to me that the primary issue was some large numbers
		being produced in one or more steps of the math underlying the model fit.
		After being uncertain how to address this in Python, I pivoted over to using
		R's \texttt{glm} package, initially did not have much luck, but ultimately
		stumbled upon the log linking function as the appropriate choice. This
		allowed the models to converge and get a proper fit.

	\section{Conclusions}

		The greatest conclusion I've drawn from this analysis is that workload is
		tremendously secondary to sleep for most intents and purposes. If we're
		carefully trying to monitor athlete loading but we aren't attending to
		sleep, we're probably not taking care of the big pillars supporting
		performance. I could see this becoming particularly relevant during
		travel, particularly where international competition is concerned.

	\section{Directions for Future Work}

		Originally, I wanted to have a pure Python analytical pipeline that performed
		a grid search across several different models and hyperparameter combinations.
		Some of these models (e.g., random forest regressors) don't produce coefficient
		estimates in the traditional sense or may not necessarily make assumptions that
		fit the data very well, but could still be very predictive. Others, like the
		multilayer perceptron, could do a great job predicting values of the
		outcome variable but might not necessarily be very readily interpretable.
		However, there are ways around this: a few solutions that come to mind are
		Shapley values, as well as visualizing outcome variable surfaces when varying
		two inputs at a time while we hold the other predictors constant. One can
		also use principal component analysis to find eigenvectors in the inputs,
		reduce the input to be two-dimensional, and then visualize changes in the
		outcome variable that way.

		Another question I would like to more extensively answer is how much of
		the variation in the wellness and workload variables simply depends upon
		player position group. Although this may not change much from a modeling
		standpoint, it could have some important practical applications, since a
		player's position could be a good heuristic for how we need to approach
		managing their workload. That being said, it could be that there are
		underlying player positions that do not correspond to the given groups,
		and discovering those could give us more insight into how to coach these
		players.

\end{document}

